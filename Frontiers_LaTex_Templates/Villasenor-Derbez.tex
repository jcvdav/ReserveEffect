\documentclass{frontiersSCNS}
\usepackage{url,hyperref,lineno,microtype,subcaption}
\usepackage[onehalfspacing]{setspace}

\linenumbers

\usepackage[utf8]{inputenc}

\def\keyFont{\fontsize{8}{11}\helveticabold }
\def\firstAuthorLast{Villasenor-Derbez {et~al.}}
\def\Authors{Juan Carlos Villaseñor-Derbez,\(^{1,*}\), Stuart Fulton,\(^{2}\) Jorge
Torre,\(^{2}\)}
% Affiliations should be keyed to the author's name with superscript numbers and be listed as follows: Laboratory, Institute, Department, Organization, City, State abbreviation (USA, Canada, Australia), and Country (without detailed address information such as city zip codes or street names).
% If one of the authors has a change of address, list the new address below the correspondence details using a superscript symbol and use the same symbol to indicate the author in the author list.
\def\Address{\(^{1}\)Bren School of Environmental Science and Management, University
of California, Santa Barbara, Santa Barbara, CA,
USA\newline \(^{2}\)Comunidad y Biodiversidad A.C., Guaymas, Mexico}
% The Corresponding Author should be marked with an asterisk
% Provide the exact contact address (this time including street name and city zip code) and email of the corresponding author
\def\corrAuthor{Me, at home}

\def\corrEmail{\href{mailto:jvillasenor@bren.ucsb.edu}{\nolinkurl{jvillasenor@bren.ucsb.edu}}}

\begin{document}
\onecolumn
\firstpage{1}

\title[Mexican marine reserves]{Three archetypes of no-take marine reserves yield positive (but
idiosyncratic) social and ecological benefits in Mexico} 

\author[\firstAuthorLast ]{\Authors} %This field will be automatically populated
\address{} %This field will be automatically populated
\correspondance{} %This field will be automatically populated

\extraAuth{}

\maketitle



\section{Introduction}\label{introduction}

La sobrepesca y prácticas pesqueras no sostenibles son unas de las
mayores amenazas para la conservación de los ecosistemas marinos del
mundo \citep{halpern_2008-dK} (Halpern et al., 2017). La implementación
de reservas marinas (\emph{i.e.} áreas donde la captura de una o más
especies está prohibida) es una medida de manejo frecuentemente
propuesta para recuperar stocks pesqueros e impulsar la productividad
pesquera en aguas cercanas (Afflerbach et al., 2014; Krueck et al.,
2017; Sala and Giakoumi, 2017). Recientes trabajos han demostrado que
también pueden mitigar y proveer amortiguamiento ante el cambio
climático (Roberts et al., 2017), variabilidad ambiental (Micheli et
al., 2012), resolver problemas de pesca incidental (Hastings et al.,
2017) y, en general, incrementar la biomasa, riqueza y densidades de
organismos dentro de sus fronteras (Giakoumi et al., 2017; Lester et
al., 2009; Sala and Giakoumi, 2017).

En México, las reservas marinas han sido comúnmente establecidas como
zonas núcleo dentro de Reservas de la Biósfera (RBs), administradas por
la Comisión Nacional de Áreas Naturales Protegidas (CONANP). Al día de
hoy, 36 RBs protegen una porción del ambiente marino en México. Sin
embargo, solamente 26 de estas incluyen (pequeñas) zonas núcleo donde
las actividades pesqueras están prohibidas. Aunque la CONANP ha hecho
esfuerzos importantes por involucrar a los actores durante la
implementación de las reservas, esto aún se caracteriza por un proceso
descendente, el cual conlleva a la falta de cumplimiento por parte de
los actores. La escasez de recursos monetarios y humanos de la limitan
también el monitoreo y vigilancia de las reservas, y a su vez, el
desempeño de la reserva.

Buscando promover una alternativa con procesos ascendentes para
implementar reservas marinas, las Organizaciones de la Sociedad Civil
(OSCs) comenzaron a trabajar con comunidades pesqueras para establecer
reservas comunitarias (Uribe et al., 2010) . Estas son comúnmente
establecidas dentro de zonas de concesión, una forma de derechos de uso
territoriales para pesquerías (TURF, en inglés). Al permitir a los
pescadores diseñar sus propias reservas, una mayor proporción de la
comunidad está de acuerdo con los perímetros y reglas establecidas, y
por lo tanto los respetan (Beger et al., 2004; Espinosa-Romero et al.,
2014; Gelcich and Donlan, 2015) . Adicionalmente, los pescadores pueden
implementar sus reservas por un periodo acordado (usualmente cinco
años), después del cual la reserva puede ser abierta a la pesca. Esto
provee a los pescadores con un sentido de confianza de que, en caso de
ser necesario, aún tienen acceso a pescar esa zona{[}\^{}1{]}. Las
reservas son directamente vigiladas y monitoreadas por la comunidad,
quienes comúnmente utilizan pequeñas embarcaciones (\emph{e.g.} pangas)
para patrullar la zona, o realizan avistamientos desde la costa en
búsqueda de pescadores ilegales Aún así, las reservas comunitarias
carecen de reconocimiento legal; por lo tanto, no hay forma de penalizar
a los infractores.

\section{Materials and Methods}\label{materials-and-methods}

\section{Results}\label{results}

\section{Discussion}\label{discussion}

\section*{References}\label{references}
\addcontentsline{toc}{section}{References}

\hypertarget{refs}{}
\hypertarget{ref-afflerbach_2014-HP}{}
Afflerbach, J. C., Lester, S. E., Dougherty, D. T., and Poon, S. E.
(2014). A global survey of -reserves, territorial use rights for
fisheries coupled with marine reserves. \emph{Global Ecology and
Conservation} 2, 97--106.
doi:\href{https://doi.org/10.1016/j.gecco.2014.08.001}{10.1016/j.gecco.2014.08.001}.

\hypertarget{ref-beger_2004-Y8}{}
Beger, M., Harborne, A. R., Dacles, T. P., Solandt, J.-L., and Ledesma,
G. L. (2004). A framework of lessons learned from community-based marine
reserves and its effectiveness in guiding a new coastal management
initiative in the philippines. \emph{Environ Manage} 34, 786--801.
doi:\href{https://doi.org/10.1007/s00267-004-0149-z}{10.1007/s00267-004-0149-z}.

\hypertarget{ref-espinosaromero_2014-PY}{}
Espinosa-Romero, M. J., Rodriguez, L. F., Weaver, A. H.,
Villanueva-Aznar, C., and Torre, J. (2014). The changing role of ngos in
mexican small-scale fisheries: From environmental conservation to
multi-scale governance. \emph{Marine Policy} 50, 290--299.
doi:\href{https://doi.org/10.1016/j.marpol.2014.07.005}{10.1016/j.marpol.2014.07.005}.

\hypertarget{ref-gelcich_2015-Gw}{}
Gelcich, S., and Donlan, C. J. (2015). Incentivizing biodiversity
conservation in artisanal fishing communities through territorial user
rights and business model innovation. \emph{Conserv Biol} 29,
1076--1085.
doi:\href{https://doi.org/10.1111/cobi.12477}{10.1111/cobi.12477}.

\hypertarget{ref-giakoumi_2017-V2}{}
Giakoumi, S., Scianna, C., Plass-Johnson, J., Micheli, F.,
Grorud-Colvert, K., Thiriet, P., et al. (2017). Ecological effects of
full and partial protection in the crowded mediterranean sea: A regional
meta-analysis. \emph{Sci Rep} 7, 8940.
doi:\href{https://doi.org/10.1038/s41598-017-08850-w}{10.1038/s41598-017-08850-w}.

\hypertarget{ref-halpern_2017-Zi}{}
Halpern, B. S., Frazier, M., Afflerbach, J., O'Hara, C., Katona, S.,
Stewart Lowndes, J. S., et al. (2017). Drivers and implications of
change in global ocean health over the past five years. \emph{PLoS ONE}
12, e0178267.
doi:\href{https://doi.org/10.1371/journal.pone.0178267}{10.1371/journal.pone.0178267}.

\hypertarget{ref-hastings_2017-sm}{}
Hastings, A., Gaines, S. D., and Costello, C. (2017). Marine reserves
solve an important bycatch problem in fisheries. \emph{Proc Natl Acad
Sci U S A}.
doi:\href{https://doi.org/10.1073/pnas.1705169114}{10.1073/pnas.1705169114}.

\hypertarget{ref-krueck_2017-J1}{}
Krueck, N. C., Ahmadia, G. N., Possingham, H. P., Riginos, C., Treml, E.
A., and Mumby, P. J. (2017). Marine reserve targets to sustain and
rebuild unregulated fisheries. \emph{PLoS Biol} 15, e2000537.
doi:\href{https://doi.org/10.1371/journal.pbio.2000537}{10.1371/journal.pbio.2000537}.

\hypertarget{ref-lester_2009-Ks}{}
Lester, S., Halpern, B., Grorud-Colvert, K., Lubchenco, J., Ruttenberg,
B., Gaines, S., et al. (2009). Biological effects within no-take marine
reserves: A global synthesis. \emph{Mar. Ecol. Prog. Ser.} 384, 33--46.
doi:\href{https://doi.org/10.3354/meps08029}{10.3354/meps08029}.

\hypertarget{ref-micheli_2012-EU}{}
Micheli, F., Saenz-Arroyo, A., Greenley, A., Vazquez, L., Espinoza
Montes, J. A., Rossetto, M., et al. (2012). Evidence that marine
reserves enhance resilience to climatic impacts. \emph{PLoS ONE} 7,
e40832.
doi:\href{https://doi.org/10.1371/journal.pone.0040832}{10.1371/journal.pone.0040832}.

\hypertarget{ref-roberts_2017-J9}{}
Roberts, C. M., O'Leary, B. C., McCauley, D. J., Cury, P. M., Duarte, C.
M., Lubchenco, J., et al. (2017). Marine reserves can mitigate and
promote adaptation to climate change. \emph{Proc Natl Acad Sci U S A}
114, 6167--6175.
doi:\href{https://doi.org/10.1073/pnas.1701262114}{10.1073/pnas.1701262114}.

\hypertarget{ref-sala_2017-69}{}
Sala, E., and Giakoumi, S. (2017). No-take marine reserves are the most
effective protected areas in the ocean. \emph{ICES Journal of Marine
Science}.
doi:\href{https://doi.org/10.1093/icesjms/fsx059}{10.1093/icesjms/fsx059}.

\hypertarget{ref-uribe_2010-u2}{}
Uribe, P., Moguel, S., Torre, J., Bourillon, L., and Saenz, A. (2010).
\emph{Implementación de reservas marinas en méxico}. 1st ed. Mexico.

\end{document}