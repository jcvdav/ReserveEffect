\begin{table}[!ht]

\caption{{\bf Variables for the Social-Ecological System analysis.}}
\centering
\resizebox{\linewidth}{!}{
\begin{tabular}{>{\raggedright\arraybackslash}p{6.5cm}|>{\raggedright\arraybackslash}p{12.2cm}}
\hline
Variable & Narrative\\
\hline
\multicolumn{2}{l}{\textbf{Resource System (RS)}}\\
\hline
\hspace{1em}RS2 - Clarity of system boundaries: Clarity of geographical boundaries of TURF and reserves & Individual TURF and reserve boundaries are explicitly outlined in official documents that include maps and coordinates. Reserve placement is decided by the community. Fishers use GPS units and landmarks.\\
\hline
\hspace{1em}RS3 - Size of resource system: TURF Area (Km$^2$) & IN = 889.5; ME = 353.1; PH = 299.7\\
\hline
\hspace{1em}RS3 - Size of resource system: Reserve area (Evaluated reserve area; Km$^2$) & IN = 2 (1.3); ME = 10.48(0.09); PH = 11.25 (4.37)\\
\hline
\hspace{1em}RS4.1 - Stock status: Status of the main fishery & Lobster stocks are well managed, and are (IN) or have been (ME, PH) MSC certified.\\
\hline
\hspace{1em}*RS5 - Age of reserves: Years since reserves were implemented & IN = 12; ME = 6; PH = 5\\
\hline
\multicolumn{2}{l}{\textbf{Resource Unit (RU)}}\\
\hline
\hspace{1em}RU1 - Resource unit mobility & Adult spiny lobsters can move between 1 and 10 Km, while larvae can have displacements in the order of hundreds of Km (Aceves-Bueno et al., 2017; Green et al., 2017).\\
\hline
\hspace{1em}RU5 - Number of units (catch diversity): Number of targeted species & Lobster is their main fishery of these three communities, but they also target finfish (2 spp each). Additionally, fishers from Isla Natividad target other sedentary benthic invertebrates (4 spp).\\
\hline
\multicolumn{2}{l}{\textbf{Actors (A)}}\\
\hline
\hspace{1em}A1 - Number of relevant actors: Number of fishers & IN = 98; ME = 80; PH = 21\\
\hline
\hspace{1em}*A3 - Isolation: Level of isolation of the fishing grounds & Their fishing grounds and reserves are highly isolated and away from dense urban centers. IN lies 545 Km south from Tijuana, and ME and PH 230 Km south from Cancun, where the nearest international airports are located.\\
\hline
\multicolumn{2}{l}{\textbf{Governance system (G)}}\\
\hline
\hspace{1em}GS6.1.4.3 - Territorial use communal rights : Presence of institutions that grant exclusive harvesting rights & Each community has exclusive access to harvest benthic resources, including lobster. These take the form of Territorial User Rights for Fisheries granted by the government to fishing cooperatives.\\
\hline
\hspace{1em}GS6.2 - Operational rules: Rules implemented by individuals atuhorized to partake on collective activities & Fishers have rules in addition to what the legislation mandates. These are: larger minimum catch sizes, lower quotas, and assigning fishers to specific fishing grounds within their TURF.\\
\hline
\hspace{1em}GS9.1 - Social monitoring: Monitoring of the activities performed by cooperative members and external fishers & Fishing cooperatives have a group (Consejo de vigilancia) that monitors and enforces formal and internal rules. They ensure fishers of their fishing cooperative adhere to the established rules, and that foreign vessels do not poach their TURF and reserves.\\
\hline
\hspace{1em}GS9.2 - Biophysical monitoring: Monitoring of biological resources, including targeted species & Fishers perform annual standardized underwater surveys in the reserves and fishing grounds. Recently, they have installed oceanographic sensors to monitor oceanographic variables.\\
\hline
GS10.1 - Graduated sanctions & Fishers have penalties for breaking collective-choice rules or fishing inside the reserves. These may range from scoldings and warnings to not being allowed to harvest a particular resource or being expelled from the cooperative.\\
\hline
\end{tabular}}
\label{table:ses}
IN = Isla Natividad, ME = Maria Elena, PH = Punta Herrero. Alphanumeric codes for variables follow \cite{basurto_2013-oq}; an asterisk (*) denotes variables incorporated based on \cite{difranco_2016-Xw} and \cite{edgar_2014-UO}. The presented narrative applies equally for all communities unless otherwise noted.
\end{table}