\documentclass{frontiersSCNS}
\usepackage{url,hyperref,lineno,microtype,subcaption}
\usepackage[onehalfspacing]{setspace}
\usepackage{float}


\linenumbers

\usepackage[utf8]{inputenc}
\floatplacement{figure}{H}

\def\keyFont{\fontsize{8}{11}\helveticabold }
\def\firstAuthorLast{Villaseñor-Derbez {et~al.}}
\def\Authors{Juan Carlos Villaseñor-Derbez\(^{1,*}\), Eréndira
Aceves-Bueno\(^{1,2}\), Stuart Fulton\(^{3}\), Álvin Suarez\(^{3}\),
Arturo Hernández-Velasco\(^{3}\), Jorge Torre\(^{3}\), Fiorenza
Micheli\(^{4}\)}
% Affiliations should be keyed to the author's name with superscript numbers and be listed as follows: Laboratory, Institute, Department, Organization, City, State abbreviation (USA, Canada, Australia), and Country (without detailed address information such as city zip codes or street names).
% If one of the authors has a change of address, list the new address below the correspondence details using a superscript symbol and use the same symbol to indicate the author in the author list.
\def\Address{\(^{1}\)Bren School of Environmental Science and Management, University
of California, Santa Barbara, Santa Barbara, CA, USA \newline
\(^{2}\)Nicholas School of the Environment, Duke University, Beaufort,
NC, USA \newline \(^{3}\)Comunidad y Biodiversidad A.C., Guaymas,
Sonora, Mexico \newline \(^{4}\)Hopkins Marine Station and Center for
Ocean Solutions, Stanford University, Pacific Grove, CA, USA}
% The Corresponding Author should be marked with an asterisk
% Provide the exact contact address (this time including street name and city zip code) and email of the corresponding author
\def\corrAuthor{Juan Carlos Villaseñor-Derbez, Bren Hall, University of California,
Santa Barbara, Santa Barbara, CA, 93106}

\def\corrEmail{\href{mailto:juancarlos@ucsb.edu}{\nolinkurl{juancarlos@ucsb.edu}}}

\begin{document}
\onecolumn
\firstpage{1}

\title[Community-based TURF-reserves]{Effectiveness of community-based TURF-reserves in Mexican small-scale
fisheries} 

\author[\firstAuthorLast ]{\Authors} %This field will be automatically populated
\address{} %This field will be automatically populated
\correspondance{} %This field will be automatically populated

\extraAuth{}

\maketitle



\begin{abstract}

Coastal marine ecosystems provide livelihoods for small-scale fishers
and coastal communities around the world. Small-scale fisheries face
great challenges since they are difficult to monitor, enforce, and
manage. Combining territorial use rights for fisheries (TURF) with
no-take marine reserves to create TURF-reserves can improve the
performance of small-scale fisheries by buffering fisheries from
environmental variability and management errors, while ensuring that
fishers reap the benefits of conservation investments. In the last 12
years, 18 old and new community-based Mexican TURF-reserves gained legal
recognition thanks to a regulation passed in 2014; their effectiveness
has not been formally evaluated. We combine causal inference techniques
and the Social-Ecological Systems framework to provide a holistic
evaluation of community-based TURF-reserves in three coastal communities
in Mexico. We find that while reserves have not yet achieved their
stated goal of increasing the density of lobster and other benthic
invertebrates, they continue to receive support from the fishing
communities. A lack of clear ecological and socioeconomic effects likely
results from a combination of factors. First, some of these reserves
might be too young for the effects to show. Second, the reserves are not
large enough to protect mobile species, like lobster. Third, variable
and extreme oceanographic conditions have impacted harvested
populations. Fourth, local fisheries are already well managed, and it is
unlikely that reserves might have a detectable effect in catches.
However, these reserves may provide a foundation for establishing
additional, larger marine reserves needed to effectively conserve mobile
species.




\medskip
\tiny
 \keyFont{ \section{Keywords:} TURF-reserves, Causal Inference, Social-Ecological Systems, Marine
Protected Areas, Marine Conservation, Small-Scale Fisheries}



\end{abstract}


\hypertarget{introduction}{%
\section{Introduction}\label{introduction}}

Marine ecosystems around the world sustain significant impacts due to
overfishing and unsustainable fishing practices
\citep{pauly_2005-qV,worm_2006-IB,halpern_2008-dK}. In particular,
small-scale fisheries face great challenges since they tend to be hard
to monitor and enforce \citep{costello_2012}. One of the many approaches
taken to improve the performance of coastal fisheries and health of the
local resources is through the implementation of Territorial Use Rights
for Fisheries (TURFs) that contain no-take marine reserves, thus
creating TURF-reserve systems
\citep{afflerbach_2014,gelcich_2015,lester_2017}.

TURFs are a fisheries management tool in which a well-defined group of
fishers (\emph{e.g.} fishing cooperatives) have exclusive access to an
explicitly delimited portion of the ocean. They promote a sense of
stewardship and incentivise resource users to sustainably manage their
resources \citep{gelcich_2008,costello_2010,mccay_2014}. On the other
hand, no-take marine reserves (marine reserves from hereinafter) are
areas where all extractive activities are off-limits. These can be
implemented to protect biodiversity but also as fishery management tools
to aid in the recovery of marine stocks. These instruments can be
combined by establishing a marine reserve within a TURF, thus making
them TURF-reserves \citep{afflerbach_2014,gelcich_2015,lester_2017}.

Conservation science has shown how marine reserves lead to increased
biomass, species richness, and abundance within the protected regions
\citep{lester_2009}, and that these may have a series of additional
benefits such as climate change mitigation, protection from
environmental variability, and fisheries benefits
\citep{roberts_2017-J9,micheli_2012-EU,krueck_2017-J1}. Likewise,
research on TURFs has shown that these areas have higher abundance of
targeted species than sites operating under open access and even similar
to that of marine reserves \citep{gelcich_2008,gelcich_2012}. The
benefits resulting from reserves established within TURFs (\emph{i.e.}
TURF-reserves) should be captured exclusively by the group of fishers
with exclusive access \citep{gelcich_2015}. Although in theory these
systems are successful \citep{smallhornwest_2018}, there is little
empirical evidence of their effectiveness and the drivers of their
success. Moreover, TURF-reserve systems are inherently intricate
social-ecological systems, and their effectiveness must depend on how
environmental and social factors combine and interact
\citep{ostrom_2009,gelcich_2015}. It is therefore important to consider
not only the indicators of interest, but also the governance settings
under which the reserves operate.

Recent norms in fisheries regulation in Mexico provide a ripe
opportunity to study the effectiveness of community-based TURF-reserves
in small-scale fisheries. In Mexico, a norm created in 2014 allows
fishers to request legal recognition of community-based reserves as
``Fish Refuges'' (\emph{Zona de Refugio Pesquero}; \citet{nom}). Since
2012, old and new marine reserves have gained legal recognition as Fish
Refuges. Of these, 18 were originally implemented within TURFs. However,
their effectiveness has not yet been formally evaluated and reported in
the scientific literature.

Here, we combine causal inference techniques and the Social-Ecological
Systems (SES) framework to evaluate community-based TURF-reserves in
three coastal communities in Mexico. The objective of this work is
twofold. First, to provide a holistic evaluation of the effectiveness of
community-based TURF-reserves in terms of the changes in biological and
socioeconomic indicators and the governance settings under which these
develop, which may inform similar processes in other countries. Second,
to identify opportunities where improvement or adjustment might lead to
increased effectiveness. We draw from lessons learned in these three
case studies and provide management recommendations to maximize the
effectiveness of community-based TURF-reserves in small-scale fisheries
where this tool is used to manage and rebuild coastal fisheries.

\hypertarget{methods}{%
\section{Methods}\label{methods}}

\hypertarget{turf-reserves-in-mexico}{%
\subsection{TURF-reserves in Mexico}\label{turf-reserves-in-mexico}}

Before discussing our data collection methods and describing our
analyses, our case studies warrant some background. Community-based
marine reserves that are implemented within TURFs are a form of
TURF-reserve, voluntarily established and enforced by local communities.
This bottom-up approach increases compliance and self-enforcement, and
reserves can yield benefits similar to systematically-designed reserves
\citep{beger_2004,smallhornwest_2018}. Community-based spatial closures
occur in different contexts, like the \emph{kapu} or \emph{ra'ui} areas
in the Pacific Islands \citep{johannes_2002,bohnsack_2004}. However,
community-based reserves can be hard to enforce if they are not legally
recognized. In such conditions, TURF fishers must rely on the exclusive
access of the TURF to maintain high levels of compliance.

In an effort to bridge this normative gap, Mexican Civil Society
Organizations (CSOs) served as a link between fishers and government,
and helped create a legal framework that solves this governance issue:
Fish Refuges \citep{nom}. Fish Refuges can be implemented as permanent,
temporary or partial reserves, which can protect one, some, or all
resources within their boundaries. One of the ways in which fishing
communities have taken advantage of this new tool is by implementing
temporary marine reserves within their TURFs with a defined expiration
date (often five years). When the expiration date is reached, fishers
can chose to open the reserves to fishing or re-establish them. Our work
focuses on Fish Refuges implemented as community-based TURF-reserves in
small-scale fisheries.

The most common setup of community-based TURF-reserves in Mexico is the
following. Fishers from a given community are assembled in fishing
cooperatives which have exclusive fishing rights over a spatially
delimited area (\emph{i.e.} TURFs shown as blue polygons in Fig
\ref{fig:map}A). Each TURF is exclusively fished by one cooperative, and
each community usually hosts no more than one cooperative. The profits
from each TURF are shared amongst all fishers from the cooperative.
Fishing cooperatives interested in implementing marine reserves work
with CSOs to implement marine reserves within their TURFs (\emph{i.e.}
TURF-reserves). Fishers then ask the government to grant legal
recognition to their TURF-reserves as Fish Refuges following a series of
studies outlined in the regulation \citep{nom}.

\hypertarget{study-areas}{%
\subsection{Study areas}\label{study-areas}}

We evaluate three community-based no-take TURF-reserve systems
implemented in Mexican TURF-managed fisheries, therefore making them
TURF-reserves (Fig \ref{fig:map}A). The first one was created by the
\emph{Buzos y Pescadores de la Baja California} fishing cooperative,
located in Isla Natividad in the Baja California Peninsula (Fig
\ref{fig:map}B). The main fishery in the island is the spiny lobster
(\emph{Panulirus interruptus}), but other resources like finfish, sea
cucumber, sea urchin, snail, and abalone are also an important source of
income. In 2006, the community decided to implement two marine reserves
within their fishing grounds. The objective of these reserves was ``to
protect commercially important invertebrate species''; mainly lobster
and abalone. The reserves obtained legal recognition in 2018
\citep{dof_website_2018}.

The other two TURF-reserve systems are located in Maria Elena and Punta
Herrero, in the Yucatan Peninsula (Fig \ref{fig:map}C). In contrast with
Isla Nativdad, which hosts a well-established fishing community, Maria
Elena is a fishing camp visited intermittently during the fishing season
that belongs to the \emph{Cozumel} fishing cooperative. Punta Herrero is
home to the \emph{José María Azcorra} fishing cooperative, and similar
to Isla Natividad hosts a small community. Their main fishery is the
Caribbean spiny lobster (\emph{Panulirus argus}), but they also target
finfish in the off-season. Maria Elena and Punta Herrero established
eight and four marine reserves in 2012 and 2013, respectively. These
reserves have been legally recognized as Fishing Refuges since their
original implementation \citep{dof_website_2012,dof_website_2013} and
subsequent re-establishments \citep{dof_website_2017b}.

These communities are representative of their region in terms of
ecology, socioeconomic, and governance aspects. Isla Natividad, for
example, is part of a greater group of fishing cooperatives belonging to
a Federation of Fishing Cooperatives. This group has been identified as
a cohesive group that cooperates to better manage their resources
\citep{mccay_2014,mccay_2017,acevesbueno_2017}. Likewise, Maria Elena
and Punta Herrero are representative of fishing cooperatives in the
Mexican Caribbean, which are also part of a regional Federation.
Together, these three communities provide an accurate representation of
other fishing communities that have been historically manged with TURFs
in each of their regions. While each region has additional communities
that have established community-based TURF-reserves, available data
would not allow us to perform the in-depth causal inference analysis
that we undertake. Yet, given the similarities among communities and the
socioeconomic and governance setting under which they operate, it is
safe to cautiously generalize our insights to other similar
community-based TURF-reserves in Mexico and elsewhere.

The regulation governing the implementation of Fish Refuges states that
these are fishery management tools intended to have biological or
socioeconomic benefits \citep{nom}. For this reason, the main portion of
our analyses focuses on a series of biological and socioeconomic
indicators that may respond to reserve implementation. However, the
effectiveness of conservation and fisheries management interventions
also depends on the social and governance structures in place. We
therefore incorporate a reduced version of the Social Ecological Systems
framework \citep{ostrom_2009} and evaluate variables and indicators
known to aid and hinder the effectiveness of management interventions in
conservation and fisheries. The incorporation of the SES is not intended
to relate different levels of governance with reserve effectiveness, but
rather help provide context on the social-ecological system in which
reserves develop. The following two sections describe our data
collection methods and analyses.

\hypertarget{data-collection}{%
\subsection{Data collection}\label{data-collection}}

We use three main sources of information to evaluate these reserves
across ecological, socioeconomic, and governance dimensions. Ecological
data come from the annual ecological monitoring of reserve and control
sites. Reserve sites are areas where no fishing occurs. Control sites
are areas that meet the following criteria: i) habitat characteristics
are similar to the corresponding reserves, ii) presumably had a similar
probability of being selected as reserves during the design phase, iii)
are located within the TURF, where fishing occurs, and iv) are not
directly adjacent to the reserves. We focus our evaluation on sites
where data are available for reserve and control sites, before and after
the implementation of the reserve. This provides us with a
Before-After-Control-Impact (\emph{i.e.} BACI) sampling design that
allows us to capture and control for temporal and spatial dynamics
\citep{stewartoaten_1986,depalma_2018} and causally attribute the
changes to the reserve
\citep{francinifilho_2008,Villasenor-Derbez_2018}.

The biological data are collected by members from each community and
personnel from the Mexican CSO \emph{Comunidad y Biodiversidad}
(\href{www.cobi.org.mx}{COBI}). Trained divers record species richness
and abundances of fish and invertebrate species along replicate
transects (30 \(\times\) 2 m each) at depths 5-20 m in the reserves and
control sites \citep{suman_2010-ez,fulton_2018,fulton_2019}. Size
structures are also collected during fish surveys. All sites were
surveyed annually, and at least once before implementation of the
reserves. A summary of sampling effort is shown in the supplementary
materials (Tables S1-S2).

Socioeconomic data come from landing receipts reported to the National
Commission for Aquaculture and Fisheries (\emph{Comisión Nacional de
Acuacultura y Pesca}; CONAPESCA). Data contain monthly lobster landings
(Kg) and revenues (MXP) for TURF-managed cooperatives with and without
marine reserves. In this case our treated unit are the cooperatives
(\emph{i.e.} communities) that have implemented a reserve within their
TURF, and the controls are nearby communities that have a TURF but did
not implement a reserve. Cooperatives incorporated in this analysis
belong to larger regional-level Cooperative Federations, and are exposed
to the same markets and institutional frameworks, making them plausible
controls \citep{mccay_2017,ayer_2018}. Landings and revenues were
aggregated at the cooperative-year level, and revenues were adjusted to
represent 2014 values by the Consumer Price Index for Mexico
\citep{oecd_2017}. A table with summary statistics for this data is
provided in the supplementary materials (Table S3, Figure S5).

Data for the evaluation of the SES were collected at the community-level
from official documents used in the design, creation, and designation of
the marine reserves. These include the technical studies that the
cooperatives submit when they request recognition of their reserves, as
well as the official enactments
\citep{dof_website_2012,dof_website_2013,dof_website_2018}. We also
complimented information based on the authors' experience and knowledge
of the communities. We collected information on the Resource Systems,
Resource Units, Actors, and Governance System (Table \ref{table:ses}).

\hypertarget{data-analysis}{%
\subsection{Data analysis}\label{data-analysis}}

We evaluate the effect that the TURF-reserves have had on four
ecological and two socioeconomic indicators shown in Table
\ref{table:indicators}. Recall that reserves were implemented to protect
lobster and other benthic invertebrates. However, we also use the
available fish and invertebrate data to test for associated co-benefits.

We use a difference-in-differences analysis to evaluate these
indicators. This approach is widely used in econometric literature to
estimate the average treatment effect of an intervention, like the
impact of minimum wage increases on employment rates \citep{card_1994}.
In our case it allows us to estimate the effect that the reserve had on
each biological and socioeconomic indicator (Table
\ref{table:indicators}) by comparing trends across time and treatments
\citep{moland_2013,Villasenor-Derbez_2018}. To perform
difference-in-differences, we regress the indicator of interest on a
dummy variable for treatment, a dummy variable for years, and the
interaction term between these with a multiple linear regression of the
form:

\begin{equation}
I_{i,t} = \alpha + \gamma_{t} Year_t + \beta Zone_i + \lambda_{t} Year_t\times Zone_i + \epsilon_{i,t}
\label{eqn:reg_bio}
\end{equation}

Where year-level fixed effects capturing a temporal trend are
represented by \(\gamma_t Year_t\), and \(\beta Zone_i\) captures the
difference between reserve (\(Zone = 1\)) and control (\(Zone = 0\))
sites. The effect of the reserve is captured by the \(\lambda_t\)
coefficient, and represents the difference observed between the control
site before the implementation of the reserve and the treated sites at
time \(t\) after controlling for other time and space variations
(\emph{i.e.} \(\gamma_t\) and \(\beta\) respectively). Therefore, we
would expect this term to be positive if the indicator increases because
of the reserve. Finally, \(\epsilon_{i,t}\) represents the error term of
the regression.

Socioeconomic indicators are evaluated with a similar approach. Due to
data constrains, we only evaluate socioeconomic data for Isla Natividad
(2000 - 2014) and Maria Elena (2006 - 2013). Neighboring communities are
used as counterfactuals that allow us to control for unobserved
time-invariants. Each focal community (\emph{i.e.} Isla Natividad and
Maria Elena) has three counterfactual communities.

\begin{equation}
I_{i,t} = \alpha + \gamma_{t} Year_t + \beta Treated_i + \lambda_{t} Year_t\times Treated_i +\epsilon_{i,t}
\label{eqn:soc_reg}
\end{equation}

The coefficient interpretations remains as for Eq. \ref{eqn:reg_bio},
but in this case the \(Treated\) dummy variable indicates if the
community has a reserve (\(Treated = 1\)) or not (\(Treated = 0\)).
These regression models allow us to establish a causal link between the
implementation of marine reserves and the observed trends by accounting
for temporal and site-specific dynamics \citep{depalma_2018}. We fit one
model per community and indicators (\emph{e.g.} there are three models
for lobster density, one for each community) for a total of 12
biological model fits and four socioeconomic model fits. Model
coefficients were estimated via ordinary least-squares and used
heteroskedastic-robust standard errors \citep{zeileis_2004-7n}. All
analyses were performed in R version 3.5.2 and R Studio version 1.1.456
\citep{R_2018}. All data and code needed to reproduce our analyses are
available in a GitHub repository at:
\url{https://github.com/jcvdav/ReserveEffect}.

We use the SES framework to evaluate each community and create a
narrative that provides context for each community. The use of this
framework standardizes our analysis and allows us to communicate our
results in a common language across fields by using a set of previously
defined variables and indicators. We based our variable selection
primarily on \citet{leslie_2015-na} and \citet{basurto_2013-oq}, who
operationalized and analyzed Mexican fishing cooperatives using this
framework, and identified the key variables relevant to fishing
cooperatives in Mexico. We also incorporate other relevant variables
known to influence reserve performance following
\citet{difranco_2016-Xw} and \citet{edgar_2014-UO}. Table
\ref{table:ses} shows the selected variables, along with definitions and
values.

\hypertarget{results}{%
\section{Results}\label{results}}

The following sections present the effect that marine reserves had on
the biological and socioeconomic indicators for each coastal community.
Results are presented in terms of difference through time and across
sites, relative to the control site on the year of implementation
(\emph{i.e.} the difference-in-differences estimate or effect size
\(\lambda_t\) from Eqs. \ref{eqn:reg_bio} and \ref{eqn:soc_reg}). We
also provide an overview of the governance settings of each community,
and discuss how these might be related to the effectiveness and
performance of the reserves.

\hypertarget{biological-effects}{%
\subsection{Biological effects}\label{biological-effects}}

Indicators showed ambiguous responses through time for each reserve.
Figure \ref{fig:indicators}A shows positive effect sizes for lobster
densities in Isla Natividad and Punta Herrero during the first years,
but the effect is eroded through time. In the case of Maria Elena,
positive changes were observed in the third and fourth year. These
effects are in the order of 0.2 extra organisms \(\mathrm{m}^{-2}\) for
Isla Natividad and Punta Herrero, and 0.01 organisms \(\mathrm{m}^{-2}\)
for Maria Elena, but are not significantly different from zero
(\(p > 0.05\)). Likewise, no significant changes were detected in fish
biomass or invertebrate and fish densities (Fig.
\ref{fig:indicators}B-D), where effect sizes oscillated around zero
without clear trends. Figures and tables with time series of indicators
and model coefficients are presented in the supplementary materials
(Figures S1-S4, Tables S4-S6).

\hypertarget{socioeconomic-effects}{%
\subsection{Socioeconomic effects}\label{socioeconomic-effects}}

Lobster landings and revenue were only available for Isla Natividad and
Maria Elena (Fig \ref{fig:lobsters}). For all years before
implementation, the effect sizes are close to zero, indicating that the
control and treatment sites have similar pre-treatment trends,
suggesting that these are plausible controls. However, effect sizes do
not change after the implementation of the reserve. Interestingly, the
negative effect observed for Isla Natividad on year 5 corresponds to the
2011 hypoxia events \citep{micheli_2012-EU}. The only positive change
observed in lobster landings is for Isla Natividad in 2014
(\(p < 0.1\)). The three years of post-implementation data for Maria
Elena do not show a significant effect of the reserve. Isla Natividad
shows higher revenues after the implementation of the reserve, as
compared to the control communities. However, these changes are only
significant for the third year (\(p < 0.05\)). Full tables with model
coefficients are presented in the supplementary materials (Tables
S4-S5).

\hypertarget{governance}{%
\subsection{Governance}\label{governance}}

Our analysis of the SES (Table \ref{table:ses}) shows that all analyzed
communities share similarities known to foster sustainable resource
management and increase reserve effectiveness. For example, fishers
operate within clearly outlined TURFs (RS2, GS6.1.4.3) that provide
exclusive access to resources and reserves. Along with their relatively
small groups (A1 - Number of relevant actors), Isolation (A3),
Operational rules (GS6.2), Social monitoring (GS9.1), and Graduated
sanctions (GS10.1), these fisheries have solid governance structures
that enable them to monitor their resources and enforce rules to ensure
sustainable management. In general, success of conservation initiatives
depends on the incentives of local communities to maintain a healthy
status of the resources upon which they depend \citep{jupiter_2017}. Due
to the clarity of access rights and isolation, the benefits of
conservation directly benefit the members of the fishing cooperatives,
which have favored the development of efficient community-based
enforcement systems. However, our SES analysis also highlights factors
that might hinder reserve performance or mask outcomes. While total
reserve size ranges from 0.2\% to 3.7\% of the TURF area, individual
reserves are often small (RS3); the largest reserve is only 4.37 km
\textsuperscript{2}, and the smallest one is 0.09 km
\textsuperscript{2}. Reserves are also relatively young (RS5).
Additionally, fishers harvest healthy stocks (RS4.1), and it is unlikely
that marine reserves will result in increased catches.

\hypertarget{discussion}{%
\section{Discussion}\label{discussion}}

Our results indicate that these TURF-reserves have not increased lobster
densities. Additionally, no co-benefits were identified when using other
ecological indicators aside from the previously reported buffering
effect that reserves can have to environmental variability in Isla
Natividad \citep{micheli_2012-EU}. The socioeconomic indicators
pertaining landings and revenues showed little to no change after
reserve implementation. Lastly, the communities exhibit all the social
enabling conditions for effective reserve and resource management. Here
we discuss possible shortcomings in our analyses as well as possible
explanations for the observed patterns.

While many ecology studies have used BACI sampling designs and
respective analyses (\emph{e.g.} \citet{stewartoaten_1986}), few
conservation studies have done so to evaluate the effect of an
intervention (\emph{e.g.}
\citet{francinifilho_2008,lester_2009,moland_2013}) which has resulted
in a call for more robust analyses in conservation science
\citep{guidetti_2002,ferraro_2006}. Our approach to evaluate the
temporal and spatial changes provides a more robust measure of reserve
effectiveness, and captures previously described patterns. For example,
the rapid increase observed for lobster densities in Isla Natividad on
the sixth year (\emph{i.e.} 2012; Fig. \ref{fig:indicators}A), occurs a
year after the hypoxia events described by \citet{micheli_2012-EU},
which caused mass mortality of sedentary organisms such as abalone and
sea urchins, but not lobster and finfish. The use of causal inference
techniques may help us support evidence-based conservation.

Our analyses of socioeconomic indicators has two limitations. First, we
only look at landings and revenues by landings for communities with and
without TURF-reserves. There are a number of other possible indicators
that could show a change due to the implementation of the reserve.
Notably, one often cited in the literature is additional benefits, such
as tourism \citep{viana_2017}. However, it is unlikely that the
evaluated communities will experience tourism benefits due to their
remoteness and the lack of proper infrastructure to sustain tourism. A
second limitation of our socioeconomic analysis is that we do not
observe effort data, which may mask the effect of the reserve. For
example, if catches remain relatively unchanged but fishing effort
decreased, that would imply a larger catch per unit effort and thus
higher profitability, provided that cost per unit effort does not
increase.

A first possible explanation for the lack of effectiveness may be the
young age of the reserves. Literature shows that age and enforcement are
important factors that influence reserve effectiveness
\citep{edgar_2014-UO,babcock_2010}. Isla Natividad has the oldest
reserves, and our SES analysis suggests that all communities have a
well-established community-based enforcement system. With these
characteristics, one would expect the reserves to be effective. Maria
Elena and Punta Herrero are relatively young reserves (\emph{i.e.}
\textless{} 6 years old) and effects may not yet be evident due to the
short duration of protection, relative to the life histories of the
protected species; community-based marine reserves in tropical
ecosystems may take six years or more to show a spillover effect
\citep{dasilva_2015-zX}.

Another key condition for effectiveness is reserve size
\citep{edgar_2014-UO}, and the lack of effectiveness can perhaps be
attributed to poor ecological coherence in reserve design (\emph{sensu}
\citet{rees_2018}). Previous research has shown that reserves in Isla
Natividad yield fishery benefits for the abalone fishery
\citep{rossetto_2015-V0}, however, abalone are less mobile than
lobsters, and perhaps the reserves provide enough protection to these
sedentary invertebrates, but not lobsters. Design principles developed
by \citet{green_2017} for marine reserves in the Caribbean state that
reserves ``should be more than twice the size of the home range of
adults and juveniles'', and suggest that reserves seeking to protect
spiny lobsters should have at least 14 km across. Furthermore, fishers
may favor implementation of reserves that pose low fishing costs due to
their small size or location. Our analysis of economic data supports
this hypothesis, as neither landings nor revenues showed the expected
short-term reductions associated to the first years of reserve
implementation \citep{ovando_2016-Wg}.

Even if reserves had appropriate sizes and were placed in optimal
locations, there are other plausible explanations for the observed
patterns. For instance, marine reserves are only likely to provide
fisheries benefits if initial population sizes are low and the fishery
is poorly managed \citep{hilborn_2004,hilborn_2006}. Both lobster
fisheries were certified by the Marine Stewardship Council and are
managed via species-specific minimum catch sizes, seasonal closures,
protection of ``berried'' females, and escapement windows where traps
are allowed \citep{dof_website_1993}. It is uncertain whether such a
well-managed fishery will experience additional benefits from marine
reserves; reserves implemented in TURFs where fishing pressure is
already optimally managed will still show a trade-off between fisheries
and conservation objectives \citep{lester_2017}. Furthermore,
\citet{gelcich_2008} have shown that TURFs alone can have greater
biomass and richness than areas operating under open access. This might
reduce the difference between indicators from the TURF and reserve
sites, making it difficult to detect such a small change. Further
research should focus on evaluating sites in the reserve, TURF, and open
access areas or similar Fish Refuges established without the presence of
TURFs where the impact of the reserves might be greater.

Finally, extreme conditions, including prolonged hypoxia, heat waves,
and storms have affected both the Pacific and Caribbean regions, with
large negative impacts on coastal marine species and ecosystems
\citep{cavole_2016,hughes_2018,breitburg_2018}. The coastal ecosystems
where these reserves are located have been profoundly affected by these
events \citep{micheli_2012-EU,woodson_2018}. Effects of protection might
be eliminated by the mortalities associated with these extreme
conditions.

While the evaluated reserves have failed to provide fishery benefits to
date, there are a number of additional ecological, fisheries, and social
benefits. Marine reserves provide protection to a wider range of species
and vulnerable habitat. Previous research focusing on these specific
sites has shown that they serve as an insurance mechanism against
uncertainty and errors in fisheries management, as well as mild
environmental shocks
\citep{micheli_2012-EU,deleo_2015,roberts_2017-J9,aalto}.
Self-regulation of fishing effort can serve as a way to compensate for
future declines associated to environmental variation
\citep{finkbeiner_2018}. Furthermore, embarking on a marine conservation
project can bring the community together, which promotes social cohesion
and builds social capital \citep{fulton_2019}. Showing commitment to
marine conservation and sustainable fishing practices has allowed
fishers to have greater bargaining power and leverage over fisheries
management \citep{prezramrez_2012}. These additional benefits might
explain why communities show a positive perception about their
performance and continue to support their presence by re-implementing
the reserves \citep{ayer_2018}.

Community-based TURF-reserves in small-scale fisheries may be helpful
conservation and fishery management tools when appropriately implemented
\citep{gelcich_2015}. We must promote bottom-up design and
implementation processes like the ones in the evaluated reserves, but
without setting design principles aside. Having full community support
surely represents an advantage, but it is important that community-based
TURF-reserves meet essential design principles such as size and
placement so as to maximize their effectiveness. Furthermore,
conservation and advocacy groups should consider the opportunity costs
of such interventions (\emph{sensu} \citet{smith_2010}) and evaluate the
potential of other approaches that may yield similar benefits.

In terms of fisheries regulation in Mexico, our work only evaluates Fish
Refuges established within TURFs. Future research should aim at
evaluating other Fish Refuges established as bottom-up processes but
without the presence of TURFs (\emph{e.g.} \citet{dof_websiteC_2012}),
others established through top-down processes (\emph{i.e.}
\citet{dof_websiteU_2018}), as well as the relationship between
governance and effectiveness across this gradient of approaches. For the
particular case of the reserves that we evaluate, the possibility of
expanding reserves or merging existing polygons into larger areas should
be evaluated and proposed to the communities.

\section*{Conflict of Interest Statement}

The authors declare that the research was conducted in the absence of
any commercial or financial relationships that could be construed as a
potential conflict of interest.

\section*{Author Contributions}

JC and AS conceived the idea. JC and EA analyzed data, discussed the
results, and wrote the first draft. FM, SF, AS, JT, and AHV discussed
the results and edited the manuscript. All authors provided valuable
contributions.

\section*{Funding}

JCVD received funding from UCMexus - CONACyT Doctoral Fellowship (CVU
669403) and the Latin American Fisheries Fellowship Program. AS, AHV, SF
and JT received funding from Marisla Foundation, Packard Foundation,
Walton Family Foundation, Summit Foundation, and Oak Foundation. FM was
supported by NSF-CNH and NSF BioOce (grants DEB-1212124 and 1736830).

\section*{Acknowledgments}

The authors wish to acknowledge Imelda Amador for contributions on the
governance data, as well as pre-processing biological data. This study
would have not been possible without the effort by members of the
fishing communities here mentioned, who participated in the
data-collection process. The authors wish to acknowledge comments by the
reviewers and editor, which significantly improved the quality of this
work.

\clearpage

\bibliographystyle{frontiersinSCNS_ENG_HUMS}
\bibliography{references}

\clearpage

\section*{Figure captions}

\begin{figure}
\centering
\includegraphics{manuscript_files/figure-latex/unnamed-chunk-7-1.pdf}
\caption{\label{fig:unnamed-chunk-7}\label{fig:map}Location of the three
coastal communities studied (A). Isla Natividad (B) is located off the
Baja California Peninsula, Maria Elena and Punta Herrero (C) are located
in the Yucatan Peninsula. Blue polygons represent the TURFs, and red
polygons the marine reserves.}
\end{figure}

\begin{figure}
\centering
\includegraphics{manuscript_files/figure-latex/unnamed-chunk-8-1.pdf}
\caption{\label{fig:unnamed-chunk-8}\label{fig:indicators}Effect sizes for
marine reserves from Isla Natividad (IN; red circles), Maria Elena (ME;
blue triangles), and Punta Herrero (PH; green squares) for lobster
densities (\emph{Panulirus spp}; A), fish biomass (B), invertebrate
densities (C), and fish densities (D). Plots are ordered by survey type
(left column: invertebrates; right column: fish). Points are jittered
hotizontally to avoid overplotting. Points indicate the effect size and
error bars are heteroskedastic-robust standard errors. Years have been
centered to year of implementation.}
\end{figure}

\begin{figure}
\centering
\includegraphics{manuscript_files/figure-latex/unnamed-chunk-9-1.pdf}
\caption{\label{fig:unnamed-chunk-9}\label{fig:lobsters}Effect sizes for
lobster catches (A) and revenues (B) in at Isla Natividad (IN; red
circles) and Maria Elena (ME; blue triangles). Points are jittered
hotizontally to avoid overplotting. Points indicate the effect size and
error bars are heteroskedastic-robust standard errors. Years have been
centered to year of implementation.}
\end{figure}

\begin{table}[H]

\caption{\label{tab:unnamed-chunk-10}\label{table:indicators}List of indicators used to evaluate the effectiveness of marine reserves, grouped by category.}
\centering
\begin{tabular}[t]{l|l}
\hline
Indicator & Units\\
\hline
\multicolumn{2}{l}{\textbf{Biological}}\\
\hline
\hspace{1em}Lobster density & org $\mathrm{m}^{-2}$\\
\hline
\hspace{1em}Invertebrate density & org $\mathrm{m}^{-2}$\\
\hline
\hspace{1em}Fish density & org $\mathrm{m}^{-2}$\\
\hline
\hspace{1em}Fish biomass & Kg $\mathrm{m}^{-2}$\\
\hline
\multicolumn{2}{l}{\textbf{Socioeconomic}}\\
\hline
\hspace{1em}Income from target species & M MXP\\
\hline
\hspace{1em}Landings from target species & Metric Tonnes\\
\hline
\end{tabular}
\end{table}

\begin{table}[H]

\caption{\label{tab:unnamed-chunk-11}\label{table:ses}Variables for the Social-Ecological System analysis (IN = Isla Natividad, ME = Maria Elena, PH = Punta Herrero). Alphanumeric codes follow \citet{basurto_2013-oq}; an asterisk (*) denotes variables incorporated based on \citet{difranco_2016-Xw} and \citet{edgar_2014-UO}. The presented narrative applies equally for all communities unless otherwise noted.}
\centering
\resizebox{\linewidth}{!}{
\begin{tabular}[t]{>{\raggedright\arraybackslash}p{6.5cm}|>{\raggedright\arraybackslash}p{12cm}}
\hline
Variable & Narrative\\
\hline
\multicolumn{2}{l}{\textbf{Resource System (RS)}}\\
\hline
\hspace{1em}RS2 - Clarity of system boundaries: Clarity of geographical boundaries of TURF and reserves & Individual TURF and reserve boundaries are explicitly outlined in official documents that include maps and coordinates. Reserve placement is decided by the community. Fishers use GPS units and landmarks.\\
\hline
\hspace{1em}RS3 - Size of resource system: TURF Area (Km$^2$) & IN = 889.5; ME = 353.1; PH = 299.7\\
\hline
\hspace{1em}RS3 - Size of resource system: Reserve area (Evaluated reserve area; Km$^2$) & IN = 2 (1.3); ME = 10.48(0.09); PH = 11.25 (4.37)\\
\hline
\hspace{1em}RS4.1 - Stock status: Status of the main fishery & Lobster stocks are well managed, and are (IN) or have been (ME, PH) MSC certified.\\
\hline
\hspace{1em}*RS5 - Age of reserves: Years since reserves were implemented & IN = 12; ME = 6; PH = 5\\
\hline
\multicolumn{2}{l}{\textbf{Resource Unit (RU)}}\\
\hline
\hspace{1em}RU5 - Number of units (catch diversity): Number of targeted species & Lobster is their main fishery of these three communities, but they also target finfish (2 spp each). Additionally, fishers from Isla Natividad target other sedentary benthic invertebrates (4 spp).\\
\hline
\multicolumn{2}{l}{\textbf{Actors (A)}}\\
\hline
\hspace{1em}A1 - Number of relevant actors: Number of fishers & IN = 98; ME = 80; PH = 21\\
\hline
\hspace{1em}*A3 - Isolation: Level of isolation of the fishing grounds & Their fishing grounds and reserves are highly isolated and away from dense urban centers. IN lies 545 Km south from Tijuana, and ME and PH 230 Km south from Cancun, where the nearest international airports are located.\\
\hline
\multicolumn{2}{l}{\textbf{Governance system (G)}}\\
\hline
\hspace{1em}GS6.1.4.3 - Territorial use communal rights : Presence of institutions that grant exclusive harvesting rights & Each community has exclusive access to harvest benthic resources, including lobster. These take the form of Territorial User Rights for Fisheries granted by the government to fishing cooperatives.\\
\hline
\hspace{1em}GS6.2 - Operational rules: Rules implemented by individuals atuhorized to partake on collective activities & Fishers have rules in addition to what the legislation mandates. These are: larger minimum catch sizes, lower quotas, and assigning fishers to specific fishing grounds within their TURF.\\
\hline
\hspace{1em}GS9.1 - Social monitoring: Monitoring of the activities performed by cooperative members and external fishers & Fishing cooperatives have a group (Consejo de vigilancia) that monitors and enforces formal and internal rules. They ensure fishers of their fishing cooperative adhere to the established rules, and that foreign vessels do not poach their TURF and reserves.\\
\hline
\hspace{1em}GS9.2 - Biophysical monitoring: Monitoring of biological resources, including targeted species & Fishers perform annual standardized underwater surveys in the reserves and fishing grounds. Recently, they have installed oceanographic sensors to monitor oceanographic variables.\\
\hline
GS10.1 - Graduated sanctions & Fishers have penalties for breaking collective-choice rules or fishing inside the reserves. These may range from scoldings and warnings to not being allowed to harvest a particular resource or being expelled from the cooperative.\\
\hline
\end{tabular}}
\end{table}



\end{document}
