\documentclass{frontiersSCNS}
\usepackage{url,hyperref,lineno,microtype,subcaption}
\usepackage[onehalfspacing]{setspace}
\usepackage{float}


\linenumbers

\usepackage[utf8]{inputenc}
\floatplacement{figure}{H}

\def\keyFont{\fontsize{8}{11}\helveticabold }
\def\firstAuthorLast{Villaseñor-Derbez {et~al.}}
\def\Authors{Juan Carlos Villaseñor-Derbez\(^{1,*}\), Eréndira
Aceves-Bueno\(^{1,2}\), Stuart Fulton\(^{3}\), Álvin Suarez\(^{3}\),
Arturo Hernández-Velasco\(^{3}\), Jorge Torre\(^{3}\), Fiorenza
Micheli\(^{4}\)}
% Affiliations should be keyed to the author's name with superscript numbers and be listed as follows: Laboratory, Institute, Department, Organization, City, State abbreviation (USA, Canada, Australia), and Country (without detailed address information such as city zip codes or street names).
% If one of the authors has a change of address, list the new address below the correspondence details using a superscript symbol and use the same symbol to indicate the author in the author list.
\def\Address{\(^{1}\)Bren School of Environmental Science and Management, University
of California, Santa Barbara, Santa Barbara, CA,
USA\newline \(^{2}\)Nicholas School of the Environment, Duke University,
Beaufort, NC, USA\newline \(^{3}\)Comunidad y Biodiversidad A.C.,
Guaymas, Sonora, Mexico\newline \(^{4}\)Hopkins Marine Station and
Center for Ocean Solutions, Stanford University, Pacific Grove, CA, USA}
% The Corresponding Author should be marked with an asterisk
% Provide the exact contact address (this time including street name and city zip code) and email of the corresponding author
\def\corrAuthor{Juan Carlos Villaseñor-Derbez, Bren Hall, University of California,
Santa Barbara, Santa Barbara, CA, 93106}

\def\corrEmail{\href{mailto:juancarlos@ucsb.edu}{\nolinkurl{juancarlos@ucsb.edu}}}

\begin{document}
\onecolumn
\firstpage{1}

\title[Community-based marine reserves]{Effectiveness of community-based marine reserves in small-scale
fisheries} 

\author[\firstAuthorLast ]{\Authors} %This field will be automatically populated
\address{} %This field will be automatically populated
\correspondance{} %This field will be automatically populated

\extraAuth{}

\maketitle



\begin{abstract}

Coastal marine ecosystems provide livelihoods for small-scale fishers
and coastal communities around the world. Small-scale fisheries face
great challenges since they are difficult to monitor, enforce, and
manage. Combining territorial user rights for fisheries (TURF) with
no-take marine reserves to create TURF-reserves can improve the
performance of small-scale fisheries by buffering fisheries from
environmental variability and management errors, while ensuring that
fishers reap the benefits of conservation investments. In the last 12
years, 18 community-based TURF-reserves gained legal recognition thanks
to a 2014 regulation; their effectiveness has not been formally
evaluated. We combine causal inference techniques and the
Social-Ecological Systems framework to provide a holistic evaluation of
community-based TURF-reserves in three coastal communities in Mexico. We
find that while reserves have not yet achieved their stated goal of
increasing the density of lobster and other benthic invertebrates, they
continue to receive significant support from the fishing communities. A
lack of clear ecological and socioeconomic effects likely results from a
combination of factors. First, local fisheries are already well managed,
and it is unlikely that reserves might have a detectable effect. Second,
some of the reserves are not large enough to protect mobile species,
like lobster. Third, some of these reserves might be too young for the
effects to show. Fourth, variable and extreme oceanographic conditions
have impacted harvested populations. However, these reserves have shaped
small-scale fishers' way of thinking about marine conservation, which
can provide a foundation for establishing additional, larger marine
reserves needed to effectively conserve mobile species.




\medskip
\tiny
 \keyFont{ \section{Keywords:} TURF-reserves, Causal Inference, Social-Ecological Systems, Marine
Protected Areas, Marine Conservation, Small-Scale Fisheries}



\end{abstract}


Words in text: 3714 \textbar{} Words in headers: 35 \textbar{} Words
outside text (captions, etc.): 192

\clearpage

\hypertarget{introduction}{%
\section{Introduction}\label{introduction}}

Marine ecosystems around the world sustain significant impacts due to
overfishing and unsustainable fishing practices
\citep{halpern_2008-dK,worm_2006-IB,pauly_2005-qV}. In particular,
small-scale fisheries face great challenges since they tend to be hard
to monitor and enforce \citep{costello_2012}. Recent research shows that
combining Territorial Use Rights for Fisheries (TURFs) with no-take
marine reserves (MRs) can greatly improve the performance of coastal
fisheries and the health of the local resources
\citep{costello_2010-Ix,lester_2017}. Commonly known as TURF-Reserves,
these systems increase the benefits of spatial access rights allowing
the maintenance of healthy resources
\citep{afflerbach_2014-HP,lester_2017}. Although in theory these systems
are successful \citep{costello_2010-Ix,smallhornwest_2018}, there is
little empirical evidence of their effectiveness and the drivers of
their success \citep{afflerbach_2014-HP,lester_2017}.

The performance of these systems depends on how environmental and social
factors combine and interact. The science of marine reserves has largely
focused on understanding the ecological effects of these areas, which
include increased biomass, species richness, and densities of organisms
within the protected regions, climate change mitigation, and protection
from environmental variability
\citep{lester_2009-Ks,giakoumi_2017-V2,sala_2017-69,roberts_2017-J9,micheli_2012-EU}.
Modelling studies show that fishery benefits of marine reserves depend
on initial stock status and the management under which the fishery
operates, as well as reserve size and the amount of larvae exported from
these \citep{hilborn_2006,krueck_2017-J1,deleo_2015}. Other research has
focused on the relationship between socioeconomic and governance
structures and reserve effectiveness
\citep{halpern_2013,lpezangarita_2014,mascia_2017-m_}. However, to our
knowledge, no studies exist that evaluate TURF-reserves from both a
social and ecological perspective. This is especially important in
social-ecological coastal systems dominated by close interactions and
feedbacks between people and natural resources \citep{ostrom_2009-hg}.

TURF-reserves can be created as community-based marine reserves,
voluntarily established and enforced by local communities. This
bottom-up approach increases compliance and self-enforcement, and
reserves can yield benefits similar to systematically-designed reserves
\citep{gelcich_2015-Gw,espinosaromero_2014-PY,beger_2004-Y8,smallhornwest_2018}.
Community-based spatial closures occur in different contexts, like the
\emph{kapu} or \emph{ra'ui} areas in the Pacific Islands
\citep{bohnsack_2004,johannes_2002}. However, MRs are difficult to
enforce if they are not legally recognized, and fishers rely on the
exclusive access granted by the TURF. In an effort to bridge this
normative gap, Mexican Civil Society Organizations (CSOs) served as a
link between fishers and government, and created a legal framework that
solves this governance issue. In Mexico, a new norm was created in 2014
allowing fishers to request the legal recognition of community-based
reserves as ``Fish Refuges'' (\emph{Zona de Refugio Pesquero};
\citet{nom}). Fish refuges can be implemented as temporal or partial
reserves, which can protect one, some, or all resources within their
boundaries. Since 2012, old and new marine reserves have gained legal
recognition as Fishing Refuges. Oh these, 18 were originally implemented
as community-based TURF-reserves. However, their effectiveness has not
yet been formally evaluated and reported in the scientific literature.

Here, we combine causal inference techniques and the Social-Ecological
Systems (SES) framework to provide a holistic evaluation of
community-based TURF-reserves in three coastal communities in Mexico.
These three case studies span a range of ecological and social
conditions representative of different regions of Mexico. The objective
of this work is twofold. First, to provide a triple bottom line
evaluation of the effectiveness of community-based marine reserves,
which may inform similar processes in other countries. Second, to
evaluate the effectiveness of TURF-reserves established as Fish Refuges
in Mexico to identify opportunities where improvement or adjustment
might lead to increased effectiveness. We draw from lessons learned in
these three case studies and provide management recommendations to
maximize the effectiveness of community-based marine reserves in
small-scale fisheries in Mexico and in other regions around the world
where this tool is used to manage and rebuild their coastal fisheries.

\hypertarget{methods}{%
\section{Methods}\label{methods}}

\hypertarget{study-area}{%
\subsection{Study area}\label{study-area}}

We evaluate three TURF-reserves in Mexico (Fig \ref{fig:map}A). The
first one was created by the \emph{Buzos y Pescadores de la Baja
California} fishing cooperative, located in Isla Natividad in the Baja
California Peninsula (Fig \ref{fig:map}B). The main fishery in the
island is the spiny lobster (\emph{Panulirus interruptus}), but other
resources like finfish, sea cucumber, read sea urchin, snail, and
abalone are also an important source of income. In 2006, the community
decided to implement two marine reserves within their fishing grounds to
protect commercially important invertebrate species; mainly lobster and
abalone. While these reserves obtained legal recognition only in 2018
\citep{dof_website_2018}, they have been well enforced since their
implementation.

The other two TURF-reserves are located in Maria Elena and Punta
Herrero, in the Yucatan Peninsula (Fig \ref{fig:map}C). In contrast with
Isla Nativdad, which hosts a well established fishing community, Maria
Elena is a fishing camp --visited intermittently during the fishing
season-- belonging to the \emph{Cozumel} fishing cooperative; Punta
Herrero is home to the \emph{José María Azcorra} fishing cooperative,
and similar to Isla Natividad hosts a local community. Their main
fishery is the Caribbean spiny lobster (\emph{Panulirus argus}), but
they also target finfish in the off-season. Maria Elena and Punta
Herrero established eight and four marine reserves in 2012 and 2013,
respectively. These reserves have been legally recognized as Fishing
Refuges since their creation \citep{dof_website_2012,dof_website_2013}.

These communities are representative of their region in terms of
ecology, socioeconomic, and governance aspects. Isla Natividad, for
example, is part of a greater group of fishing cooperatives belonging to
a Federation of Fishing Cooperatives. This group has been identified as
a cohesive group that often cooperates to better manage their resources
\citep{mccay_2014-CN,mccay_2017-1m,acevesbueno_2017}. Likewise, Maria
Elena and Punta Herrero are representative of fishing cooperatives in
the Mexican Caribbean, which are also part of a regional Federation.
Together, these three communities provide an accurate representation of
other fishing communities in each of their regions. While each region
has additional communities that have established community-based
TURF-reserves, available data would not allow us to perform the in-depth
causal inference analysis that we undertake. Yet, given the similarities
among communities and the socioeconomic and governance setting under
which they operate, it is safe to cautiously generalize our insights to
other similar reserves in Mexico and elsewhere around the world.

\hypertarget{data-collection}{%
\subsection{Data collection}\label{data-collection}}

We use three main sources of information to evaluate these reserves
across the ecological, socioeconomic, and governance dimensions.
Ecological data come from the annual ecological monitoring of reserve
and control areas, carried out by members from each community and
personnel from the Mexican CSO \emph{Comunidad y Biodiversidad}
(\href{www.cobi.org.mx}{COBI}). Trained divers record richness and
abundances of fish and invertebrate species along replicate transects
(30 \(\times\) 2 m each) at depths 5-20 m in the reserves and control
sites \citep{fulton_2018,fulton_2019,suman_2010-ez}. Size structures are
also collected during fish surveys. We define control sites as regions
with habitat characteristics similar to the corresponding reserves, and
that presumably had a similar probability of being selected as reserves
during the design phase. We focus our evaluation on sites where data are
available for reserve and control sites, before and after the
implementation of the reserve. This provides us with a
Before-After-Control-Impact (\emph{i.e.} BACI) sampling design that
allows us to capture and control for temporal and spatial dynamics
\citep{depalma_2018,ferraro_2006-oW}. BACI designs and causal inference
techniques have proven effective to evaluate marine reserves, as they
allow us to causally attribute observed changes to the intervention
\citep{moland_2013-VP,Villasenor-Derbez_2018}. All sites were surveyed
annually, and at least once before implementation of the reserves.

Socioeconomic data come from landing receipts reported to the National
Commission for Aquaculture and Fisheries (\emph{Comisión Nacional de
Acuacultura y Pesca}; CONAPESCA). Data contain monthly lobster landings
(Kg) and revenues (MXP) for cooperatives with and without marine
reserves. Cooperatives incorporated in this analysis belong to larger
regional-level Cooperative Federations, and are exposed to the same
markets and institutional frameworks, making them plausible controls
\citep{mccay_2017-1m,ayer_2018}. Landings and revenues were aggregated
at the cooperative-year level, and revenues were adjusted to represent
2014 values by the Consumer Price Index for Mexico \citep{oecd_2017-VV}.

Data for the evaluation of the SES were collected at the community-level
from official documents used in the creation and designation of the
marine reserves
\citep{dof_website_2012,dof_website_2013,dof_website_2018} and based on
the authors' experience and knowledge of the communities. These include
information on the Resource Systems, Resource Units, Actors, and
Governance System (Table \ref{table:ses}).

\hypertarget{data-analysis}{%
\subsection{Data analysis}\label{data-analysis}}

We evaluate the effect that marine reserves have had on four ecological
and two socioeconomic indicators (Table \ref{table:indicators}). Recall
that reserves were implemented to protect lobster and other benthic
invertebrates. However, we also use the available fish data to test for
associated co-benefits.

We use a difference-in-differences analysis to evaluate these
indicators. This approach allows us to estimate the effect that the
reserve had by comparing trends across time and treatments
\citep{moland_2013-VP,Villasenor-Derbez_2018}. The analysis of
ecological indicators is performed with a multiple linear regression of
the form:

\begin{equation}
I_{i,t,j} = \alpha + \gamma_{t} Year_t + \beta Zone_i + \lambda_{t} Year_t\times Zone_i + \sigma_jSpp_j + \epsilon_{i,t,j}
\label{eqn:reg_bio}
\end{equation}

Where year-level fixed effects are represented by \(\gamma_t Year_t\),
and \(\beta Zone_i\) captures the difference between reserve
(\(Zone = 1\)) and control (\(Zone = 0\)) sites. The interaction term
\(\lambda_{it} Year_t\times Zone_i\) represents the mean change in the
indicator inside the reserve, for year \(t\), with respect to the year
of implementation in the control site. When evaluating biomass and
densities of the invertebrate or fish communities, we include
\(\sigma_j\) to control for species-level fixed effects.
\(\epsilon_{i,t,j}\) represents the error term of the regression.

Socioeconomic indicators are evaluated with a similar approach. Due to
data constrains, we only evaluate socioeconomic data for Isla Natividad
(2000 - 2014) and Maria Elena (2006 - 2013). Neighboring communities are
used as counterfactuals that allow us to control for unobserved
time-invariants. Each focal community (Isla Natividad and Maria Elena)
has three counterfactual communities.

\begin{equation}
I_{i,t,j} = \alpha + \gamma_{t} Year_t + \beta Treated_i + \lambda_{t} Year_t\times Treated_i + \sigma_jCom_j +\epsilon_{i,t,j}
\label{eqn:soc_reg}
\end{equation}

The model interpretation remains as for Eq \ref{eqn:reg_bio}, but in
this case the \(Treated\) dummy variable indicates if the community has
a reserve (\(Treated = 1\)) or not (\(Treated = 0\)) and \(\sigma_jCom\)
captures community-level fixed-effects. These regression models allow us
to establish a causal link between the implementation of marine reserves
and the observed trends by accounting for temporal and spatial dynamics
\citep{depalma_2018}. The effect of the reserve is captured by the
\(\lambda_t\) coefficient, and represents the difference observed
between the control site before the implementation of the reserve and
the treated sites at time \(t\) after controlling for other time and
space variations (\emph{i.e.} \(\gamma_t\) and \(\beta\) respectively).
All model coefficients were estimated via ordinary least-squares and
heteroskedastic-robust standard errors \citep{zeileis_2004-7n}. All
analyses were performed in R version 3.5.1 (2018-07-02) and R Studio
version 1.1.456 \citep{R_2018}.

We use the SES framework to evaluate each community. The use of this
framework standardizes our analysis and allows us to communicate our
results in a common language across fields by using a set of previously
defined variables and indicators. We based our variable selection
primarily on \citet{leslie_2015-na} and \citet{basurto_2013-oq}, who
operationalized and analyzed Mexican fishing cooperatives using this
framework. We also incorporate other relevant variables known to
influence reserve performance following \citet{difranco_2016-Xw} and
\citet{edgar_2014-UO}. Table \ref{table:ses} shows the selected
variables, their definition and values.

\hypertarget{results}{%
\section{Results}\label{results}}

The following sections present the effect that marine reserves had on
each of the biological and socioeconomic indicators for each coastal
community. Results are presented in terms of the difference through time
and across sites, relative to the control site on the year of
implementation (\emph{i.e.} effect size \(\lambda_t\)). We also provide
an overview of the governance settings of each community, and discuss
how these might be related to the effectiveness and performance of the
reserves.

\hypertarget{biological-effects}{%
\subsection{Biological effects}\label{biological-effects}}

Indicators showed ambiguous responses through time for each reserve.
Figure \ref{fig:indicators}A shows positive effect sizes for lobster
densities in Isla Natividad and Punta Herrero during the first years,
but the effect is eroded through time. In the case of Maria Elena,
positive changes were observed in the third and fourth year. These
effects are in the order of 0.2 extra organisms \(\mathrm{m}^{-2}\) for
Isla Natividad and Punta Herrero, and 0.01 organisms \(\mathrm{m}^{-2}\)
for Maria Elena, but are not significantly different from zero
(\(p > 0.05\)). Likewise, no significant changes were detected in fish
biomass or invertebrate and fish densities (Fig.
\ref{fig:indicators}B-D), where effect sizes oscillated around zero
without clear trends. Full tables with model coefficients are presented
in the supplementary materials (S1 Table, S2 Table, S3 Table).

\hypertarget{socioeconomic-effects}{%
\subsection{Socioeconomic effects}\label{socioeconomic-effects}}

Lobster landings and revenue were only available for Isla Natividad and
Maria Elena (Fig \ref{fig:lobsters}). For all years before
implementation, the effect sizes are close to zero, indicating that the
control and treatment sites have similar pre-treatment trends,
suggesting that these are plausible controls. However, effect sizes do
not change after the implementation of the reserve. Interestingly, the
negative effect observed for Isla Natividad on year 5 correspond to the
2011 hypoxia events. The only positive change observed in lobster
landings is for Isla Natividad in 2014 (\(p < 0.1\)). The three years of
post-implementation data for Maria Elena do not show a significant
effect of the reserve. Isla Natividad shows higher revenues after the
implementation of the reserve, as compared to the control communities.
However, these changes are not significant and are associated with
increased variation. Full tables with model coefficients are presented
in the supplementary materials (S4 Table, S5 Table).

\hypertarget{governance}{%
\subsection{Governance}\label{governance}}

Our analysis of the SES (Table \ref{table:ses}) shows that all analyzed
communities share similarities known to foster sustainable resource
management and increase reserve effectiveness. For example, fishers
operate within clearly outlined TURFs (RS2, GS6.1.4.3) that provide
exclusive access to resources and reserves. Along with their relatively
small groups (A1 - Number of relevant actors), Isolation (A3),
Operational rules (GS6.2), Social monitoring (GS9.1), and Graduated
sanctions (GS10.1), these fisheries have solid governance structures
that enable them to monitor their resources and enforce rules to ensure
sustainable management. In general, success of conservation initiatives
depends on the incentives of local communities to maintain a healthy
status of the resources upon which they depend \citep{jupiter_2017}. Due
to the clarity of access rights and isolation, the benefits of
conservation directly benefit the members of the fishing cooperatives,
which have favored the development of efficient community-based
enforcement systems. However, our SES analysis also highlights factors
that might hinder reserve performance or mask outcomes. While total
reserve size ranges from 0.2\% to 3.7\% of the TURF area, individual
reserves are often small (RS3), and relatively young (RS5).
Additionally, fishers harvest healthy stocks (RS4.1), and it's unlikely
that marine reserves will result in increased catches.

\hypertarget{discussion}{%
\section{Discussion}\label{discussion}}

Our results indicate that these TURF-reserves have not increased lobster
densities. Additionally, no co-benefits were identified when using other
ecological indicators aside from the previously reported buffering
effect that reserves can have to environmental variability in Isla
Natividad \citep{micheli_2012-EU}. The socioeconomic indicators
pertaining landings and revenues showed little to no change after
reserve implementation. Despite the lack of evidence of the
effectiveness of these reserves, most of the communities show a positive
perception about their performance and continue to support their
presence \citep{ayer_2018}. Understanding the social-ecological context
in which these communities and their reserves operate might provide
insights as to why this happens.

Some works evaluate marine reserves by performing inside-outside
\citep{guidetti_2014-8Z,friedlander_2017-oI,rodriguez_2017-PD} or
before-after comparisons \citep{betti_2017-lq}. The first approach does
not address temporal variability, and the second can not distinguish
between the temporal trends in a reserve and the entire system
\citep{depalma_2018}. Our approach to evaluate the temporal and spatial
changes provides a more robust measure of reserve effectiveness. For
example, we capture previously described patterns like the rapid
increase observed for lobster densities in Isla Natividad on the sixth
year (\emph{i.e.} 2012; Fig. \ref{fig:indicators}A), a year after the
hypoxia events described by \citet{micheli_2012-EU}, which caused mass
mortality of sedentary organisms such as abalone and sea urchins, but
not lobster and finfish. Yet, our empirical approach assumes control
sites are a plausible counterfactual for treated sites. This implies
that treated sites would have followed the same trend as control sites,
had the reserves not been implemented. Nonetheless, temporal trends for
each site don't show any significant increases (S1 Table, S2 Table, S3
Table), supporting our findings of lack of change in the indicators
used.

A first possible explanation for the lack of effectiveness may be the
young age of the reserves. Literature shows that age and enforcement are
important factors that influence reserve effectiveness
\citep{edgar_2014-UO,babcock_2010}. Isla Natividad has the oldest
reserves, and our SES analysis suggests that all communities have a
well-established community-based enforcement system. With these
characteristics, one would expect the reserves to be effective. Maria
Elena and Punta Herrero are relatively young reserves (\emph{i.e.}
\textless{} 6 years old) and effects may not yet be evident due to the
short duration of protection, relative to the life histories of the
protected species; community-based marine reserves in tropical
ecosystems may take six years or more to show a spillover effect
\citep{dasilva_2015-zX}.

Another key condition for effectiveness is reserve size
\citep{edgar_2014-UO}, and the lack of effectiveness can perhaps be
attributed to poor ecological coherence in reserve design (\emph{sensu}
\citet{rees_2018}). Previous research has shown that reserves in Isla
Natividad yield fishery benefits for the abalone fishery
\citep{rossetto_2015-V0}. Abalone are less mobile than lobsters, and
perhaps the reserves provide enough protection to these sedentary
invertebrates, but not lobsters. Design principles developed by
\citet{green_2017} for marine reserves in the Caribbean state that
reserves ``should be more than twice the size of the home range of
adults and juveniles'', and suggest that reserves seeking to protect
spiny lobsters should have at least 14 km across. Furthermore, fishers
may favor implementation of reserves that pose low fishing costs due to
their small size or location. Our analysis of economic data supports
this hypothesis, as neither landings nor revenues showed the expected
short-term costs associated to the first years of reserve implementation
\citep{ovando_2016-Wg}.

Even if reserves had appropriate sizes and were placed in optimal
locations, there are other plausible explanations for the observed
patterns. For instance, marine reserves are only likely to provide
fisheries benefits if initial population sizes are low and the fishery
is poorly managed \citep{hilborn_2004,hilborn_2006}. Both lobster
fisheries were certified by the Marine Stewardship Council
\citep{prezramrez_2016-J1}. Additionally, lobster fisheries are managed
via species-specific minimum catch sizes, seasonal closures, protection
of ``berried'' females, and escapement windows where traps are allowed
\citep{dof_website_1993}. It is uncertain whether such a well-managed
fishery will experience additional benefits from marine reserves.
Furthermore, \citet{gelcich_2008} have shown that TURFs alone can have
greater biomass and richness than areas operating under open access.
This might reduce the difference between indicators from the TURF and
reserve sites, making it difficult to detect such a small change.
Further research should focus on evaluating sites in the reserve, TURF,
and open access areas or similar Fish Refuges established without the
presence of TURFs where the impact of the reserves might be larger.

Finally, extreme conditions, including prolonged hypoxia, heat waves,
and storms have affected both the Pacific and Caribbean regions, with
large negative impacts of coastal marine species and ecosystems
\citep{cavole_2016,hughes_2018,breitburg_2018}. The coastal ecosystems
where these reserves are located have been profoundly affected by these
events \citep{micheli_2012-EU,woodson}. Effects of protection might be
eliminated by the mortalities associated with these extreme conditions.

While the evaluated reserves have failed to provide fishery benefits up
to now, there are a number of additional ecological, fisheries, and
social benefits. Marine reserves provide protection to a wider range of
species and vulnerable habitat. These sites can serve as an insurance
against uncertainty and errors in fisheries management, as well as mild
environmental shocks
\citep{micheli_2012-EU,deleo_2015,roberts_2017-J9,aalto}.
Self-regulation of fishing effort (\emph{i.e.} reduction in harvest) can
serve as a way to compensate for future declines associated to
environmental variation \citep{finkbeiner_2018}. Embarking in a marine
conservation project can bring the community together, which promotes
social cohesion and builds social capital \citep{fulton_2019}. Showing
commitment to marine conservation and sustainable fishing practices
allows fishers to have greater bargaining power and leverage over
fisheries management \citep{prezramrez_2012}. Furthermore, the lack of
effectiveness observed in these reserves should not be generalizable to
other reserves established under the same legal framework (\emph{i.e.}
Fish Refuges) in Mexico, and future research should aim at evaluating
other areas that have also been established as bottom-up processes but
without the presence of TURFs (\emph{e.g.} \citet{dof_websiteC_2012}),
or others established through a top-down process (\emph{i.e.}
\citet{dof_websiteU_2018}).

Community-based marine reserves in small-scale fisheries can be helpful
conservation and fishery management tools when appropriately
implemented. Lessons learned from these cases can guide implementation
of community-based marine reserves elsewhere. For the particular case of
the marine reserves that we evaluate, the possibility of expanding
reserves or merging existing polygons into larger areas should be
evaluated and proposed to the communities. Community-based marine
reserves might have more benefits that result from indirect effects of
the reserves, particularly providing resilience to shocks and management
errors, and promoting social cohesion, which should be taken into
account when evaluating the outcomes of TURF-reserves. Having full
community support surely represents an advantage, but it is important
that community-based TURF-reserves meet essential design principles such
as size and placement so as to maximize their effectiveness.

\section*{Conflict of Interest Statement}

The authors declare that the research was conducted in the absence of
any commercial or financial relationships that could be construed as a
potential conflict of interest.

\section*{Author Contributions}

JC and AS conceived the idea. JC and EA analyzed data, discussed the
results, and wrote the first draft. FM, SF, AS, JT, and AHV discussed
the results and edited the manuscript. All authors provided valuable
contributions.

\section*{Funding}

JCVD received funding from UCMexus - CONACyT Doctoral Fellowship (CVU
669403) and the Latin American Fisheries Fellowship Program. AS, AHV, SF
and JT received funding from Marisla Foundation, Packard Foundation,
Walton Family Foundation, Summit Foundation, and Oak Foundation. FM was
supported by NSF-CNH and NSF BioOce (grants DEB-1212124 and 1736830).

\section*{Acknowledgments}

The authors wish to acknowledge Imelda Amador for contributions on the
governance data, as well as pre-processing biological data. This study
would have not been possible without the effort by members of the
fishing communities here mentioned, who participated in the
data-collection process.

\clearpage

\bibliographystyle{frontiersinSCNS_ENG_HUMS}
\bibliography{references}

\clearpage

\section*{Figure captions}

\begin{figure}
\centering
\includegraphics{Villasenor-Derbez_files/figure-latex/unnamed-chunk-7-1.pdf}
\caption{\label{fig:unnamed-chunk-7}\label{fig:map}Location of the three
coastal communities studied (A). Isla Natividad (B) is located off the
Baja California Peninsula, Maria Elena and Punta Herrero (C) are located
in the Yucatan Peninsula. Blue polygons represent the TURFs, and red
polygons the marine reserves.}
\end{figure}

\begin{figure}
\centering
\includegraphics{Villasenor-Derbez_files/figure-latex/unnamed-chunk-8-1.pdf}
\caption{\label{fig:unnamed-chunk-8}\label{fig:indicators}Effect sizes for
marine reserves from Isla Natividad (IN; red cirlcles), Maria Elena (ME;
blue triangles), and Punta Herrero (PH; green squares) for lobster
densities (\emph{Panulirus spp}; A), fish biomass (B), invertebrate
densities (C), and fish densities (D). Plots are ordered by survey type
(left column: invertebrates; right column: fish). Points are jittered
hotizontally to avoid overplotting. Points indicate the effect size and
standard errors. Years have been centered to year of implementation.}
\end{figure}

\begin{figure}
\centering
\includegraphics{Villasenor-Derbez_files/figure-latex/unnamed-chunk-9-1.pdf}
\caption{\label{fig:unnamed-chunk-9}\label{fig:lobsters}Effect sizes for
lobster catches (A) and revenues (B) in at Isla Natividad (IN; red
circles) and Maria Elena (ME; blue triangles). Points indicate the
effect size and standard errors. Years have been centered to year of
implementation.}
\end{figure}

\begin{table}[H]

\caption{\label{tab:unnamed-chunk-10}\label{table:indicators}List of indicators used to evaluate the effectiveness of marine reserves, grouped by category.}
\centering
\begin{tabular}[t]{l|l}
\hline
Indicator & Units\\
\hline
\multicolumn{2}{l}{\textbf{Biological}}\\
\hline
\hspace{1em}Lobster density & org $\mathrm{m}^{-2}$\\
\hline
\hspace{1em}Invertebrate density & org $\mathrm{m}^{-2}$\\
\hline
\hspace{1em}Fish density & org $\mathrm{m}^{-2}$\\
\hline
\hspace{1em}Fish biomass & Kg $\mathrm{m}^{-2}$\\
\hline
\multicolumn{2}{l}{\textbf{Socioeconomic}}\\
\hline
\hspace{1em}Income from target species & M MXP\\
\hline
\hspace{1em}Landings from target species & Metric Tonnes\\
\hline
\end{tabular}
\end{table}

\begin{table}[H]

\caption{\label{tab:unnamed-chunk-11}\label{table:ses}Variables for the Social-Ecological System analysis (IN = Isla Natividad, ME = Maria Elena, PH = Punta Herrero). Alphanumeric codes follow \citet{basurto_2013-oq}; an asterisk (*) denotes variables incorporated based on \citet{difranco_2016-Xw} and \citet{edgar_2014-UO}.}
\centering
\resizebox{\linewidth}{!}{
\begin{tabular}[t]{>{\raggedright\arraybackslash}p{6.5cm}|>{\raggedright\arraybackslash}p{12cm}}
\hline
Variable & Narrative\\
\hline
\multicolumn{2}{l}{\textbf{Resource System (RS)}}\\
\hline
\hspace{1em}RS2 - Clarity of system boundaries: Clarity of geographical boundaries of TURF and reserves & Individual TURF and reserve boundaries are explicitly outlined in official documents that include maps and coordinates. Reserve placement is decided by the community. Fishers use GPS units and landmarks.\\
\hline
\hspace{1em}RS3 - Size of resource system: TURF Area (Km$^2$) & IN = 889.5; ME = 353.1; PH = 299.7\\
\hline
\hspace{1em}RS3 - Size of resource system: Reserve area (Evaluated reserve area; Km$^2$) & IN = 2 (1.3); ME = 10.48(0.09); PH = 11.25 (4.37)\\
\hline
\hspace{1em}RS4.1 - Stock status: Status of the main fishery & Lobster stocks are well managed, and are (IN) or have been (ME, PH) MSC certified.\\
\hline
\hspace{1em}*RS5 - Age of reserves: Years since reserves were implemented & IN = 12; ME = 6; PH = 5\\
\hline
\multicolumn{2}{l}{\textbf{Resource Unit (RU)}}\\
\hline
\hspace{1em}RU5 - Number of units (catch diversity): Number of targeted species & Lobster is their main fishery of these three communities, but they also target finfish. Additionally, fishers from Isla Natividad target other sedentary benthic invertebrates.\\
\hline
\multicolumn{2}{l}{\textbf{Actors (A)}}\\
\hline
\hspace{1em}A1 - Number of relevant actors: Number of fishers & IN = 98; ME = 80; PH = 21\\
\hline
\hspace{1em}*A3 - Isolation: Level of isolation of the fishing grounds & Their fishing grounds and reserves are highly isolated and away from dense urban centers.\\
\hline
\multicolumn{2}{l}{\textbf{Governance system (G)}}\\
\hline
\hspace{1em}GS6.1.4.3 - Territorial use communal rights : Presence of institutions that grant exclusive harvesting rights & Each community has exclusive access to harvest benthic resources, including lobster. These take the form of Territorial User Rights for Fisheries granted by the government to fishing cooperatives.\\
\hline
\hspace{1em}GS6.2 - Operational rules: Rules implemented by individuals atuhorized to partake on collective activities & Fishers have rules in addition to what the legislation mandates. These include larger minimum catch sizes, lower quotas, and assigning fishers to specific fishing grounds within their TURF.\\
\hline
\hspace{1em}GS9.1 - Social monitoring: Monitoring of the activities performed by cooperative members and external fishers & Fishing cooperatives have a group that monitors and enforces formal and internal rules. They ensure fishers of their fishing cooperative adhere to the established rules, and that foreign vessels do not poach their TURF and reserves.\\
\hline
\hspace{1em}GS9.2 - Biophysical monitoring: Monitoring of biological resources, including targeted species & Fishers perform annual standardized underwater surveys in the reserves and fishing grounds. Recently, they have installed oceanographic sensors to monitor oceanographic variables.\\
\hline
GS10.1 - Graduated sanctions & Fishers have penalties for breaking collective-choice rules or fishing inside the reserves. These may range from scoldings and warnings to not being allowed to harvest a particular resource or being expelled from the cooperative.\\
\hline
\end{tabular}}
\end{table}



\end{document}
